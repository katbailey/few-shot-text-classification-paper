\section{Experiments}
We evaluate the accuracy of performing one-shot classification on some publicly available labeled datasets.

\subsection{Datasets}
We created 2- and 3-category subsets of the "20 Newsgroups" dataset. The original dataset consists of 20000 messages taken from 20 newsgroups on subjects like cars, religion, guns, and baseball. We used the scikit\_learn python library to extract subsets of this data. We also created 2-, 3- and 4-category subsets of the DBPedia dataset, which contains the first paragraph of the Wikipedia page for around half a million entities in 15 categories, e.g. Animal, Film, Plant, Company, etc.

We perform some basic cleaning of the text and remove stop words before converting documents into vectors.

\begin{table}[]
\centering
\captionsetup{position=bottom}
\begin{tabular}{lll}
\toprule
Categories                  & \# docs & Max acc. \\
\midrule
Village, Film                & 9307         & 0.9992       \\
Village,Animal              & 8960         & 0.9979       \\
Animal, Film                 & 8947         & 0.9946       \\
Animal, Company              & 9220         & 0.9943       \\
Animal, Film, Company, Village & 18527        & 0.9706       \\
autos, baseball              & 1046         & 0.9703       \\
guns, hardware               & 1078         & 0.9693       \\
mideast, electronics         & 1063         & 0.9689       \\
Animal, Film, Company         & 13867        & 0.9688       \\
christian, guns              & 1099         & 0.9380       \\
med, electronics             & 1112         & 0.9324       \\
atheism, space               & 1004         & 0.9232       \\
baseball, hockey             & 1069         & 0.9063       \\
autos, baseball, space        & 1605         & 0.8964       \\
Animal, Plant                & 8730         & 0.8540       \\
politics, religion           & 774          & 0.8264      \\
\bottomrule
\end{tabular}
\caption{Maximum achieved accuracy using one-shot classification}\label{bruteforce}
\end{table}

\subsection{Maximum One-Shot Accuracy}
In order to know how accurate our classification approach can be in theory (i.e. if the user happens to be lucky enough to find the best representatives for each category) we did some brute force testing of the approach, where we went through thousands of combinations of document representatives to see what the highest achievable accuracy was with one-shot learning. In some cases, i.e. where there were only two categories and only around 500 documents per category, we did an exhaustive search, testing all possible combinations. In other cases, we randomly sampled from all possible combinations. Table \ref{bruteforce} shows the maximum accuracy we achieved on various subsets of categories.

As expected the accuracy is high when the categories are very distinct, such as the "Animal" and "Film" categories in the DBPedia dataset: there would be very little overlap in the types of words used in articles about animals vs articles about films. We looked at the few misclassifications that arose when the best-known representatives were used in this dataset and found that the films misclassified as animals were films about animals and the animals misclassified as films were all thoroughbred racehorses that had won or been nominated for awards.

These results show us that for a batch of documents to be classified into separate categories, if those categories truly are reflected in the words of the documents and good enough representatives are chosen, it is possible to achieve very high accuracy with just a single labeled example per category. The problem, of course is how to choose the representatives.

\subsection{Results with LDA}
We ran LDA on our datasets specifying the number of known categories as the number of topics and then tested the maximum accuracy that could be obtained using one-shot learning when trying only the possible combinations of representatives in the top 12 documents as surfaced by LDA and described in section \ref{approach}. The idea is that the user will be presented with documents for classification, perhaps 12 documents per page, and ideally they wouldn't have to look beyond the first, or at most the second page of documents in order to choose representatives for each category.

\begin{table}[]
\centering
\captionsetup{position=bottom}
\begin{tabular}{lll}
\toprule
Categories                  & Max acc. & LDA max \\
\midrule
Village, Film                & 0.9992       & 0.998388    \\
Village, Animal              & 0.9979       & --         \\
Animal, Film                 & 0.9946       & 0.992510    \\
Animal, Company              & 0.9943       & 0.987633    \\
Animal, Film, Company, Village & 0.9706       & --         \\
autos, baseball              & 0.9703       & 0.945402    \\
guns, hardware               & 0.9693       & 0.965613    \\
mideast, electronics         & 0.9689       & 0.959472    \\
Animal, Film, Company         & 0.9688       & 0.961771    \\
christian, guns              & 0.9380       & 0.925251    \\
med, electronics             & 0.9324       & 0.931532    \\
atheism, space               & 0.9232       & 0.864271    \\
baseball, hockey             & 0.9063       & 0.798500    \\
autos, baseball, space        & 0.8964       & 0.803371    \\
Animal, Plant                & 0.8540       & --         \\
politics, religion           & 0.8264       & 0.804404   \\
\bottomrule
\end{tabular}
\caption{Maximum brute-force accuracy obtained and maximum accuracy obtained using only combinations of the top 12 LDA-surfaced documents}\label{ldaacc}
\end{table}

In table \ref{ldaacc} we show the same maximum accuracy obtained via the brute-force testing of combinations along with the maximum accuracy achieved by testing the combinations from the top 12 documents. In many cases the LDA accuracy is quite close to the maximum brute-force accuracy, however, in 3 cases we could not get a result for LDA because the top 12 documents did not include representatives from all categories. This is perhaps understandable in the case of the 4-category dataset, "Animal,Film,Company,Village." In the case of the "Animal,Plant" dataset we see that the brute force accuracy is relatively low, due to the fact that many similar words would be used to describe animals and plants ("species", "family", "habitat", etc), and so this perhaps explains why LDA had a difficult time teasing apart the topics. However in the case of the "Village,Animal" dataset, which gets a maximum brute-force accuracy of 99.79\% the same argument clearly cannot be made. This tells us that there is much room for improvement in our method for inferring topics in datasets for the purpose of surfacing good category representatives.

\begin{table}[]
\centering
\captionsetup{position=bottom}
\begin{tabular}{lll}
\toprule
Categories                      & GloVe LDA       & FastText LDA \\
\midrule
Village, Film                               & 0.9984       & 0.9932                 \\
Animal, Film                                & 0.9925       & 0.9934                 \\
MeansOfTransportation, NaturalPlace          & 0.9922       & 0.9926                 \\
Village, Company                            & 0.9895       & 0.9896                 \\
Animal, Company                             & 0.9876       & 0.9905                 \\
Building, NaturalPlace                      & 0.9791       & 0.9807                 \\
Album, Film                                 & 0.974        & 0.9825                 \\
guns, hardware                              & 0.9656       & 0.9638                 \\
Film, Company                               & 0.9632       & 0.9664                 \\
Animal, Film, Company                        & 0.9618       & 0.9445                 \\
Artist, Athlete                             & 0.9604       & 0.922                  \\
mideast, electronics                        & 0.9595       & 0.9746                 \\
autos, baseball                             & 0.9454       & 0.9511                 \\
autos, hockey                               & 0.9413       & 0.9376                 \\
Building, EducationalInstitution            & 0.9345       & 0.935                  \\
Film, WrittenWork                           & 0.9341       & 0.9376                 \\
med, electronics                            & 0.9315       & 0.9153                 \\
christian, guns                             & 0.9253       & 0.9243                 \\
med, religion                               & 0.888        & 0.888                  \\
atheism, space                              & 0.8643       & 0.8713                 \\
autos, guns                                 & 0.8417       & 0.8407                 \\
autos, electronics                          & 0.8291       & 0.845                  \\
space, med                                  & 0.8111       & 0.8405                 \\
politics, religion                          & 0.8044       & 0.7526                 \\
autos, baseball, space                       & 0.8034       & 0.8414                 \\
baseball, hockey                            & 0.7985       & 0.8594                 \\
religion, mideast                           & 0.7526       & 0.7808                 \\
christian, atheism                          & 0.6553       & 0.6777                 \\
atheism, religion                           & 0.6304       & 0.6522                 \\
\bottomrule
\end{tabular}
\caption{Max accuracies on combinations of top 12 LDA documents: GloVe vs FastText}\label{lda_fasttext}
\end{table}


We also tested the LDA accuracy on some datasets for which we had not run a brute-force test. Table \ref{lda_fasttext} shows these results, tested using both GloVe and FastText pre-trained word embeddings.

\subsection{Relative length of category representatives}
We were curious about the extent to which relative length of the chosen category representatives mattered. For example, if the user chose a very short document as the sole representative for category A and a very long document, or multiple documents, for category B, how much would this skew the predictions towards category B. On one dataset only, the 2-category subset "autos, baseball" from 20 Newsgroups, we went through every possible combination of category representatives, noted the length of each document, ran the auto-classification step, and noted the number of predictions in each category. With this information, we could look at the correlation between relative representative length and relative prediction count for that category. We found a strong positive correlation between them, roughly 0.8.
 