\documentclass{article} % For LaTeX2e
\usepackage[utf8]{inputenc} % allow utf-8 input
\usepackage[T1]{fontenc} % use 8-bit T1 fonts
\usepackage[letterpaper]{geometry}
\usepackage{latexsym}
\usepackage[linesnumbered,ruled]{algorithm2e}
\usepackage{multirow}
\usepackage{fstc,times}
\usepackage{url}
\usepackage{amssymb}
\usepackage{amsmath}
\usepackage{graphicx}
\usepackage{booktabs}
\usepackage{caption}
\captionsetup[table]{
    labelsep=newline,
    justification=centering
    }
\usepackage{subcaption}
\makeatletter
\newcommand{\@BIBLABEL}{\@emptybiblabel}
\newcommand{\@emptybiblabel}[1]{}
\makeatother
% \usepackage[draft]{hyperref}
\usepackage[hidelinks]{hyperref}


\title{Few-Shot Text Classification with Pre-trained Word Embeddings and a Human in the Loop}


\author{Katherine Bailey \and Sunny Chopra \\
  Acquia \\
  \texttt{\{katherine.bailey,sunny.chopra\}@acquia.com}
}
\date{}


\begin{document}
\maketitle

\begin{abstract}
Most of the literature around text classification treats it as a supervised learning problem: given a corpus of labeled documents, train a classifier such that it can accurately predict the classes of unseen documents. In industry, however, it is not uncommon for a business to have entire corpora of documents where few or none have been classified, or where existing classifications have become meaningless. With web content, for example, poor taxonomy management can result in labels being applied indiscriminately, making filtering by these labels unhelpful. Our work aims to make it possible to classify an entire corpus of unlabeled documents using a human-in-the-loop approach, where the content owner manually classifies just one or two documents per category and the rest can be automatically classified. This "few-shot" learning approach requires rich representations of the documents such that those that have been manually labeled can be treated as prototypes, and automatic classification of the rest is a simple case of measuring the distance to prototypes. This approach uses pre-trained word embeddings, where documents are represented using a simple weighted average of constituent word embeddings. We have tested the accuracy of the approach on existing labeled datasets and provide the results here. We have also made code available for reproducing the results we got on the 20 Newsgroups dataset\footnote{\url{https://github.com/katbailey/few-shot-text-classification}}.
\end{abstract}

\section{Introduction}

\subsection*{Word Embeddings}

Word Embeddings are representations of words as low-dimensional vectors of real numbers that capture the semantic relationships between words. In Natural Language Processing (NLP), some method for converting words to numeric values is always necessary as computation only works with numbers, not raw text. \citep{mikolov2013distributed} introduced efficient techniques for learning distributed vector representations of words from huge corpora. This method is called word2vec, and since then alternative approaches have been put forward by others such as \citep{pennington2014glove}, known as GloVe, and \citep{bojanowski2016subword}, known as FastText. There is also doc2vec for learning representations of entire documents.

These approaches differ in the way the representations are generated.  Word2vec and doc2vec are "predictive" models, whereas GloVe is categorized as a "count-based" model.  Predictive models learn their vectors by predicting the target word given neighboring context words using a feed-forward neural network. The weights of this network are optimized by using stochastic gradient descent, and these weights are vector representation of the words in the vocabulary. In contrast, count-based models learn their vectors by finding a lower dimensional representation of each word by minimizing a reconstruction loss function on the co-occurrence count matrix given as an input.

\subsection*{Few-Shot Learning}
Few-shot learning is an approach to classification that works with only a few human labeled examples. It often goes hand-in-hand with transfer learning, a technique involving learning representations during one task that can then be applied to a different task, because it is the richness of the learned representations that makes it possible to learn from just a few examples. The use of pre-trained word embeddings is an example of transfer learning. One-Shot learning is a special case of Few-Shot Learning: as the name suggests, it means learning to classify objects when only one labeled example exists per class.

\subsection*{Human-in-the-Loop}
The term Human-in-the-Loop (HitL) refers to any Machine Learning technique that involves human input in the training process. It includes techniques such as active learning, where humans handle low confidence predictions, as well as crowd-sourced approaches to labeling data sets. In our case, the humans perform the "few shots" in our few-shot learning system.

\section{Few-Shot Text Classification with a Human in the Loop}
Our approach involves a "classification engine" that the user (the content owner) interacts with. A batch of documents is fed to the engine and each document is converted into a 300-dimensional vector. A set of two or more categories is specified and then Latent Dirichlet Allocation is run on the batch using the number of categories provided as the number of topics. This serves to surface the most likely representative documents for each category. These are then presented to the user, who must choose some documents to manually classify for each category. The system then has a vector representing each category: if only one document was classified for a category then that document's vector is used as the category vector, otherwise the vectors of multiple documents are averaged together. Once this is done, the remaining documents are compared against each category representative using simple cosine similarity and each one is assigned the category whose vector it is closest to. A score is also assigned for each prediction. All of these steps are explained in section \ref{approach}.

\section{Related work}
In \citep{arora2017asimple} the authors present an approach to representing sentences or entire documents as vectors. They first take the weighted average of the constituent (pre-trained) word vectors of the document. The weighting method serves to down-weight frequent words. They then run principal components analysis (PCA) on the batch, and the final embedding for each document is obtained by subtracting the projection of the set of sentence embeddings to their first principal component. The intuition behind this is that common methods, such as GloVe, for computing word vectors based on co-occurrence statistics lead to large components that contain no semantic information. A similar insight is presented in \citep{mu2017allbuttop}.

In \citep{snell2017prototypical}, the authors present the idea of \textit{Prototypical Networks} as a way of performing few-shot classification. In this approach, each class prototype is the mean vector of the vectorized support points belonging to its class and they are learned through gradient-descent-based training episodes on subsets of training examples. Classification then involves finding the nearest class prototype for each query vector. This is similar to how we perform classification but in our case we are actively seeking the best class prototypes by using a human-in-the-loop to choose them.

\section{Approach in Detail} \label{approach}
In this section, we provide the details of the text classification system.

\subsection{Document Embeddings}
 For any given collection of documents, which we refer to as a batch, each document needs to be converted to a fixed-length vector so that we can measure similarity between them. 

For our task, which is about classifying content, we found there to be little improvement in accuracy when running the PCA step, as suggested by \newcite{arora2017asimple} and \newcite{mu2017allbuttop}, over just using the weighted average of the word vectors. Morevover, PCA introduced a level of complexity that meant the embedding of a document was always batch-specific. For pragmatic reasons it was preferable for us to have document representations that were independent of the batch they came from. Hence our embedding method is simply:

\begin{algorithm}[!h]
\SetKwInOut{Input}{Input}
\SetKwInOut{Output}{Output}
\Input{Word representations $\{v_w : w\in \mathcal{V}\}$, a set of documents $ \mathcal{D}$, parameter $\alpha$, and estimated probabilities  $\{p(w) : w\in \mathcal{V}\}$ of the words}
\Output{Document embeddings $\{v_d : d\in \mathcal{D}\}$.}
\For {all documents d in $\mathcal{D}$}{
	$ v_d \leftarrow \frac{1}{|d|} \sum_{w\in d} \frac{\alpha}{\alpha + p(w)}v_w$
	}

\caption{Weighted average algorithm for document representations.}
\label{algo:representation}
\end{algorithm}

\subsection{Classification}
The classification task involves manually labelling a small number of documents in each class and using these as representatives of the class. In the one-shot case, this means that our representation of a class is simply the single document vector that has been labeled for that class. Once we have a representative for each class, we can proceed to predict classes for the rest.

\begin{algorithm}[!h]
\SetKwInOut{Input}{Input}
\SetKwInOut{Output}{Output}
\Input{Class representative vectors $\{v_c : c\in \mathcal{C}\}$, a set of document vectors $\{v_d : d\in \mathcal{D}\}$}
\Output{Predicted class for each document.}
\For {all documents d in $\mathcal{D}$}{
	\For {all classes c in $\mathcal{C}$} {
	$ sims_c \leftarrow cosine\_similarity(v_d, v_c)$
	}
	$\hat{c_d} \leftarrow argmax(sims)$
}
\caption{Predicting classes using cosine similarity with class representative vectors.}
\label{algo:classification}
\end{algorithm}

If we have more than one representative document per class, we simply take the average of those vectors as our category representative.

\subsection{Surfacing good representatives}
Choosing good class representatives is of vital importance to the accuracy of this approach. The human-in-the-loop will be making the final choice about which documents to use as representatives for each class, but we need to make it as easy as possible to choose the best possible representatives. Our approach is to run Latent Dirichlet Allocation (LDA) on the batch of documents, assign each document to the topic it has the highest probability of belonging to, and then rank the documents within each topic in descending order of their probability of belonging to the topic. The idea is to have an ordering of the documents within a batch such that the first page of documents seen in the UI is likely to have a mix of good representatives for each topic.


\section{Experiments}
We evaluate the accuracy of performing one-shot classification on a range of datasets taken from publicly available labeled datasets.

\subsection{Datasets}

\subsection{Maximum One-Shot Accuracy}
We needed to know how accurate this approach can be in theory, if the user happens to be lucky enough to find the best representatives for each category.

\begin{table}[]
\centering
\caption{Maximum achieved accuracy using one-shot classification}
\label{my-label}
\begin{tabular}{lll}
\toprule
Categories                  & \# Documents & Max accuracy \\
\midrule
Village,Film                & 9307         & 0.9992       \\
Village,Animal              & 8960         & 0.9979       \\
Animal,Film                 & 8947         & 0.9946       \\
Animal,Company              & 9220         & 0.9943       \\
Animal,Film,Company,Village & 18527        & 0.9706       \\
autos,baseball              & 1046         & 0.9703       \\
guns,hardware               & 1078         & 0.9693       \\
mideast,electronics         & 1063         & 0.9689       \\
Animal,Film,Company         & 13867        & 0.9688       \\
christian,guns              & 1099         & 0.9380       \\
med,electronics             & 1112         & 0.9324       \\
atheism,space               & 1004         & 0.9232       \\
baseball,hockey             & 1069         & 0.9063       \\
autos,baseball,space        & 1605         & 0.8964       \\
Animal,Plant                & 8730         & 0.8540       \\
politics,religion           & 774          & 0.8264      
\end{tabular}
\end{table}

We also tested the accuracy using Fasttext and Word2Vec embeddings just on those sets of representatives that gave us the best accuracy using GloVe embeddings. It should be noted that these embeddings may well have achieved better results using different representatives.

\begin{table}[]
\centering
\caption{Accuracy achieved on the same representatives using FastText embeddings}
\label{my-label-2}
\begin{tabular}{lll}
\toprule
Categories                  & GloVe & Fasttext \\
\midrule
Village,Film                & 0.9992        & 0.9977       \\
Village,Animal              & 0.9979        & 0.9887       \\
Animal,Film                 & 0.9946        & 0.869       \\
Animal,Company              & 0.9943        & 0.8954       \\
Animal,Film,Company,Village & 0.9706        & 0.886       \\
autos,baseball              & 0.9703        & 0.6456       \\
guns,hardware               & 0.9693        & 0.7342       \\
mideast,electronics         & 0.9689        & 0.9274       \\
Animal,Film,Company         & 0.9688        & 0.812       \\
christian,guns              & 0.9380        & 0.7265       \\
med,electronics             & 0.9324        & 0.7568       \\
atheism,space               & 0.9232        & 0.6946       \\
baseball,hockey             & 0.9063        & 0.6317       \\
autos,baseball,space        & 0.8964        & 0.6454      \\
Animal,Plant                 & 0.8540        & 0.6717       \\
politics,religion           & 0.8264        & 0.7681     
\end{tabular}
\end{table}

\subsection{Finding Good Representatives}



\section{LDA Results}

 

\section{Discussion and future work}
From the experiments that we conducted on the 20 Newsgroups and DBPedia datasets, it is clear that if the user chooses good representatives for each category, we can achieve high accuracy in the document categorization task. This implies that the crucial step in our method is to present the user with suitable candidates to be labeled as representatives.  We used Latent Dirichlet Allocation (LDA) for topic inference and found that while in many cases we got close to the maximum accuracy achieved through brute-force trials, in some case it fell far short or even failed to tease apart the topics at all.

An alternative to LDA would be to use probabilistic Latent Semantic Analysis (pLSA) which treats topics as word distributions and uses probabilistic methods similar to LDA. But with Dirichlet priors for the document-topic and topic-word distributions in LDA to prevent over-fitting, producing better results, this seems like a better choice. Our goal in the future is to improve the topic inference step, and so we will look to other alternatives. One idea is to use guided LDA as suggested in \citep{conf/eacl/JagarlamudiDU12}.  The seeds for guiding it will be the category name that the user decides to classify the batch of documents.  Another approach is to use Gaussian LDA as proposed in \citep{conf/acl/DasZD15}.  This approach is a good fit as it uses word embeddings too and we use the Glove word embeddings for the content classification by representing each document as a weighted average of the word embeddings.  A novel approach is to use ProdLDA as recommended in \citep{2017arXiv170301488S}, which is a neural network version in which the distribution over individual words is a product of experts rather than the mixture model used in LDA.  Even clustering algorithms on the word embeddings representation of documents can be applied to do topic inferencing which we leave it for future work.

Although we only tested our approach on single labeled datasets, we would like to be able to apply it to multi-labeled datasets. The idea is to use each label independently and the user will select the best representative for each label. By treating each label independently of the other labels, we convert it into a single labeled problem. The challenge is to recommend good candidate documents for being a representative for each label, and that's where standard LDA will fail as it will always suggest the same documents for being good candidates of being representatives to the user for all the labels.  That's where guiding the LDA with the seeds of category names can solve this problem.  An alternative approach opposed to the standard method of treating each label independently is to use classifier chains for multi-label classification as advised in \citep{Read2009}.  This method can help us achieve a fair improvement in the accuracy.  The foremost thing again is the topic inferencing which could boost the accuracy even for the multi-label classification, and that's going to be our principal focus of research in the future.


% \clearpage
\bibliography{fstc}
\bibliographystyle{iclr2017_conference}

\end{document}


