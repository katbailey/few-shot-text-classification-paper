\documentclass{article} % For LaTeX2e
\usepackage[utf8]{inputenc} % allow utf-8 input
\usepackage[T1]{fontenc} % use 8-bit T1 fonts
\usepackage[letterpaper]{geometry}
\usepackage{latexsym}
\usepackage[linesnumbered,ruled]{algorithm2e}
\usepackage{multirow}
\usepackage{fstc,times}
\usepackage{url}
\usepackage{amssymb}
\usepackage{amsmath}
\usepackage{graphicx}
\usepackage{booktabs}
\usepackage{caption}
\captionsetup[table]{
    labelsep=newline,
    justification=centering
    }
\usepackage{subcaption}
\makeatletter
\newcommand{\@BIBLABEL}{\@emptybiblabel}
\newcommand{\@emptybiblabel}[1]{}
\makeatother
% \usepackage[draft]{hyperref}
\usepackage[hidelinks]{hyperref}


\title{Few-Shot Text Classification with Pre-trained Word Embeddings and a Human in the Loop}


\author{Katherine Bailey \and Sunny Chopra \\
  Acquia \\
  \texttt{\{katherine.bailey,sunny.chopra\}@acquia.com}
}
\date{}


\begin{document}
\maketitle

\begin{abstract}
Most of the literature around text classification treats it as a supervised learning problem: given a corpus of labeled documents, train a classifier such that it can accurately predict the classes of unseen documents. In industry, however, it is not uncommon for a business to have entire corpora of documents where few or none have been classified, or where existing classifications have become meaningless. With web content, for example, poor taxonomy management can result in labels being applied indiscriminately, making filtering by these labels unhelpful. Our work aims to make it possible to classify an entire corpus of unlabeled documents using a human-in-the-loop approach, where the content owner manually classifies just one or two documents per category and the rest can be automatically classified. This "few-shot" learning approach requires rich representations of the documents such that those that have been manually labeled can be treated as prototypes, and automatic classification of the rest is a simple case of measuring the distance to prototypes. This approach uses pre-trained word embeddings, where documents are represented using a simple weighted average of constituent word embeddings. We have tested the accuracy of the approach on existing labeled datasets and provide the results here. We have also made the code available for reproducing the results we got on the 20 Newsgroups dataset\footnote{\url{https://github.com/katbailey/few-shot-text-classification}}.
\end{abstract}

\section{Introduction}

\subsection*{Word Embeddings}

Word Embeddings are representations of words as low-dimensional vectors of real numbers that capture the semantic relationships between words. In Natural Language Processing (NLP), some method for converting words to numeric values is always necessary as computation only works with numbers, not raw text. \citep{mikolov2013distributed} introduced efficient techniques for learning distributed vector representations of words from huge corpora. This method is called word2vec, and since then alternative approaches have been put forward by others such as \citep{pennington2014glove}, known as GloVe, and \citep{bojanowski2016subword}, known as FastText. There is also doc2vec for learning representations of entire documents.

These approaches differ in the way the representations are generated.  Word2vec and doc2vec are "predictive" models, whereas GloVe is categorized as a "count-based" model.  Predictive models learn their vectors by predicting the target word given neighboring context words using a feed-forward neural network. The weights of this network are optimized by using stochastic gradient descent, and these weights are vector representation of the words in the vocabulary. In contrast, count-based models learn their vectors by finding a lower dimensional representation of each word by minimizing a reconstruction loss function on the co-occurrence count matrix given as an input.

\subsection*{Few-Shot Learning}
Few-shot learning is an approach to classification that works with only a few human labeled examples. It often goes hand-in-hand with transfer learning, a technique involving learning representations during one task that can be applied to a different task, because it is the richness of the learned representations that makes it possible to learn from just a few examples. The use of pre-trained word embeddings is an example of transfer learning. One-Shot learning is a special case of Few-Shot Learning: as the name suggests, it means learning to classify objects when only one labeled example exists per class.

\subsection*{Human-in-the-Loop}
The term Human-in-the-Loop (HitL) refers to any Machine Learning technique that involves human input in the training process. It includes techniques such as active learning, where humans handle low confidence predictions, as well as crowd-sourced approaches to labeling data sets. In our case, the humans perform the "few shots" in our few-shot learning system.

\section{Few-Shot Text Classification with a Human in the Loop}
Our approach involves a "classification engine" that the user (the content owner) interacts with. A batch of documents is fed to the engine, and each document is converted into a 300-dimensional vector. A set of two or more categories is specified, and then Latent Dirichlet Allocation is run on the batch using the number of categories provided as the number of topics. This serves to surface the most likely representative documents for each category. These are then presented to the user, who must choose some documents to manually classify for each category. The system then has a vector representing each category: if only one document was classified for a category then that document's vector is used as the category vector, otherwise, the vectors of multiple documents are averaged together. Once this is done, the remaining documents are compared against each category representative using simple cosine similarity and each one is assigned the category whose vector it is closest to. A score is also assigned for each prediction. All of these steps are explained in section \ref{approach}.

\section{Related work}
In \citep{arora2017asimple} the authors present an approach to representing sentences or entire documents as vectors. They first take the weighted average of the constituent (pre-trained) word vectors of the document. The weighting method serves to down-weight frequent words. They then run principal components analysis (PCA) on the batch, and the final embedding for each document is obtained by subtracting the projection of the set of sentence embeddings to their first principal component. The intuition behind this is that common methods, such as GloVe, for computing word vectors based on co-occurrence statistics lead to large components that contain no semantic information. A similar insight is presented in \citep{arora2017asimple}.

\section{Approach in Detail}
In this section, we provide the details of the text classification system.



\section{Experiments}
We evaluate the accuracy of performing one-shot classification on a range of datasets taken from publicly available labeled datasets.

\subsection{Datasets}

\subsection{Maximum One-Shot Accuracy}
We needed to know how accurate this approach can be in theory, if the user happens to be lucky enough to find the best representatives for each category.

\begin{table}[]
\centering
\caption{Maximum achieved accuracy using one-shot classification}
\label{my-label}
\begin{tabular}{lll}
\toprule
Categories                  & \# Documents & Max accuracy \\
\midrule
Village,Film                & 9307         & 0.9992       \\
Village,Animal              & 8960         & 0.9979       \\
Animal,Film                 & 8947         & 0.9946       \\
Animal,Company              & 9220         & 0.9943       \\
Animal,Film,Company,Village & 18527        & 0.9706       \\
autos,baseball              & 1046         & 0.9703       \\
guns,hardware               & 1078         & 0.9693       \\
mideast,electronics         & 1063         & 0.9689       \\
Animal,Film,Company         & 13867        & 0.9688       \\
christian,guns              & 1099         & 0.9380       \\
med,electronics             & 1112         & 0.9324       \\
atheism,space               & 1004         & 0.9232       \\
baseball,hockey             & 1069         & 0.9063       \\
autos,baseball,space        & 1605         & 0.8964       \\
Animal,Plant                & 8730         & 0.8540       \\
politics,religion           & 774          & 0.8264      
\end{tabular}
\end{table}

We also tested the accuracy using Fasttext and Word2Vec embeddings just on those sets of representatives that gave us the best accuracy using GloVe embeddings. It should be noted that these embeddings may well have achieved better results using different representatives.

\begin{table}[]
\centering
\caption{Accuracy achieved on the same representatives using FastText embeddings}
\label{my-label-2}
\begin{tabular}{lll}
\toprule
Categories                  & GloVe & Fasttext \\
\midrule
Village,Film                & 0.9992        & 0.9977       \\
Village,Animal              & 0.9979        & 0.9887       \\
Animal,Film                 & 0.9946        & 0.869       \\
Animal,Company              & 0.9943        & 0.8954       \\
Animal,Film,Company,Village & 0.9706        & 0.886       \\
autos,baseball              & 0.9703        & 0.6456       \\
guns,hardware               & 0.9693        & 0.7342       \\
mideast,electronics         & 0.9689        & 0.9274       \\
Animal,Film,Company         & 0.9688        & 0.812       \\
christian,guns              & 0.9380        & 0.7265       \\
med,electronics             & 0.9324        & 0.7568       \\
atheism,space               & 0.9232        & 0.6946       \\
baseball,hockey             & 0.9063        & 0.6317       \\
autos,baseball,space        & 0.8964        & 0.6454      \\
Animal,Plant                 & 0.8540        & 0.6717       \\
politics,religion           & 0.8264        & 0.7681     
\end{tabular}
\end{table}

\subsection{Finding Good Representatives}



\section{LDA Results}

 

\section{Discussion}
Our testing on the 20 Newsgroups and DBPedia datasets showed that the general approach is sound, that the document embeddings using weighted averages of pre-trained word embeddings are useful representations, and that if good category representatives are chosen, high levels of accuracy can be achieved on the classification task. It also showed that LDA can help in choosing good representatives. The datasets we tested on were quite artificial: mostly just 2-category datasets chosen from larger datasets with more categories. However they did help identify certain characteristics that may make a real dataset suitable to use this approach on. The most important characteristic is that the categories must be sufficiently distinct and the words used in the documents should reflect those categories. The length of the documents chosen as representatives is also important - they should be roughly equal in length.

As to the question of working with more categories, why not test on the entire 20 Newsgroups or DBPedia dataset? Brute-force testing becomes infeasible with larger numbers of categories, and so it wasn't possible to get a maximum achievable accuracy for the entire dataset (the maximum achieved accuracy of 97\% on a particular 4-category subset of the DBPedia dataset was based on testing 390,625 of the 625 trillion possible combinations). In any case, it is unclear whether this approach will be appropriate for datasets with many categories --- say, more than four or five --- seeing as it requires the human in the loop to find good representatives for each one, a task which might prove daunting in such a case even if the category breakdown has been successfully approximated through LDA or similar. We feel that our text classification approach is best suited to 2- or 3-category classification tasks. An interesting use case for 2-category, i.e. binary, classification might be topic stance detection. Given user-generated content on a particular topic, e.g. posts on a web forum, it might be the case that different words are used depending on where the writer stands on that topic. An example of this would be some people using the term "family reunification" and others using "chain migration" to refer to the same thing. In this case our approach could be used to detect the stance reflected in each post.

\section{Future work}
The crucial step in our method is to present the user with suitable candidates to be labeled as representatives. We used Latent Dirichlet Allocation (LDA) for topic inference and found that while in many cases we got close to the maximum accuracy achieved through brute-force trials, in some cases it fell far short or even failed to tease apart the topics at all.

An alternative to LDA would be to use probabilistic Latent Semantic Analysis (pLSA) which treats topics as word distributions and uses probabilistic methods similar to LDA. But the use of Dirichlet priors in LDA for the document-topic and topic-word distributions in order to prevent over-fitting seems to make it a better choice. Our goal in the future is to improve the topic inference step, and so we will look to other alternatives. One idea is to use guided LDA as suggested in \citep{conf/eacl/JagarlamudiDU12}.  The seeds for guiding it will be the category names that the user specifies to classify the batch of documents.  Another approach is to use Gaussian LDA as proposed in \citep{conf/acl/DasZD15}. This approach is a good fit as it works with vectorized representations of words and documents. Yet another option would be to use ProdLDA as described in \citep{2017arXiv170301488S}, which is a neural network version in which the distribution over individual words is a product of experts rather than the mixture model used in LDA. Running clustering algorithms on the document representations is another approach we plan to try in order to solve the topic inference problem.

Although we only tested our approach on single labeled datasets, we would like to be able to apply it to multi-labeled datasets. One idea would be to use each label independently and do a binary classification of whether a document has that label or not. By treating each label independently of the other labels, we convert it into a single labeled problem. However, our current LDA approach to topic inference will not work in this case as it will always suggest the same ordering of documents regardless of which label the user is choosing representatives for. This is where guiding the LDA with the seeds of category names can prove immensely useful. An alternative approach, rather than treating each label independently, would be to use classifier chains for multi-label classification as suggested in \citep{Read2009}. This method may help us achieve a fair improvement in accuracy. The foremost issue again is the topic inference, which will be essential for accurate multi-label classification, and that's going to be our principal focus of research in the future.

% \clearpage
\bibliography{fstc}
\bibliographystyle{iclr2017_conference}

\end{document}


